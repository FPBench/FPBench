\documentclass[main.tex]{subfiles}
\begin{document}

\begin{abstract}

  We introduce \name, a standard benchmark format for validation and
  optimization of numerical accuracy in floating-point
  computations. \name is a first step toward addressing an increasing
  need in our community for comparisons and combinations of tools from
  different application domains. To this end, \name provides a basic
  floating-point benchmark format and several quality measures for
  comparing different tools. Programs expressed in the \name format
  can also serve as a common intermediate representation to enable the
  combination of different tools. We describe the \name format and
  measures and show that \name expresses benchmarks from recent papers
  in the literature, by building an initial benchmark suite drawn from
  these papers.  We intend for \name to grow into a standard benchmark
  suite for the members of the floating-point tools research
  community.

%%  In this article, we propose a standard benchmark for the validation and
%%  optimization of the numerical accuracy of floating-point computations to
%%  facilitate the comparison of the tools developed in this domain.  This
%%  benchmarks responds to an increasing need of the community, to compare
%%  different tools, associated to various domains of application.  More
%%  specifically, we consider a standard floating-point benchmark format and
%%  we propose then several measures.  These measures make it possible to
%%  compare tools for the verification and improvement of the numerical
%%  accuracy of programs.  In this way, we provide a set of experimentations
%%  applied on various examples to show the useful of our suite.

\end{abstract}

\end{document}
