\documentclass[main.tex]{subfiles}
\begin{document}

\begin{abstract}

  In this article we propose FPBench, a standard benchmark format for
  validation and optimization of numerical accuracy in floating-point
  computations.  FPBench is a first step toward addressing an increasing
  need in our community for comparisons and combinations of tools from
  different application domains.  To this end, FPBench provides a basic
  floating-point benchmark format and several quality measures which enable
  comparisons across different tools. Furthermore, programs expressed in the
  FPBench format can serve as a common intermediate representation to enable the
  combination of different tools. We describe the FPBench format and
  measures and show that FPBench can already express several benchmarks
  from recent papers in the literature.  In the future we will develop a
  full standard benchmark suite in FPBench and additional features that
  further enable FPBench to serve as a common language for members of the
  research community to cooperate and compare their floating point tools.

%%  In this article, we propose a standard benchmark for the validation and
%%  optimization of the numerical accuracy of floating-point computations to
%%  facilitate the comparison of the tools developed in this domain.  This
%%  benchmarks responds to an increasing need of the community, to compare
%%  different tools, associated to various domains of application.  More
%%  specifically, we consider a standard floating-point benchmark format and
%%  we propose then several measures.  These measures make it possible to
%%  compare tools for the verification and improvement of the numerical
%%  accuracy of programs.  In this way, we provide a set of experimentations
%%  applied on various examples to show the useful of our suite.

\end{abstract}

\end{document}
