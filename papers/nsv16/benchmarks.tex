\documentclass[main.tex]{subfiles}
\begin{document}

\section{Benchmarks}
\label{sec:benchmarks}

The \name suite includes \nbenches benchmarks
  sourced from recent papers in floating point verification,
  optimization, and accuracy improvement.
Of the benchmarks,
  28 are drawn from the Herbie test suite~\cite{pavel15},
  9 from a series of papers by Martel et~al.~\cite{fmics15},
  7 from the test suite for Rosa~\cite{DarulovaK14},
  and one from the FPTaylor test suite~\cite{fptaylor-fm15}.
Together, these programs come from
  a variety of domains and applications,
  from control software to mathematical libraries
  (see Table~\ref{tbl:domains}).
The included benchmarks use the range of functionality
  available in \name (see Table~\ref{tbl:features}).

\begin{table}[hbtp]
  \begin{subtable}[t]{.5\linewidth}
    \begin{tabular}{lr}
      Name & Benchmarks \\\hline
      Basic Arithmetic & 44 \\
      Exponentials & 13 \\
      Trigonometric & 10 \\
      Comparison & 12 \\
      Conditionals & 3 \\
      Loops & 12
    \end{tabular}
    \caption{Functions and language features used in the \name
      benchmarks. Exponential functions include logarithms, the
      exponential function, and the power function.}
    \label{tbl:features}
  \end{subtable}
  \hfill
  \begin{subtable}[t]{.4\linewidth}
    \begin{tabular}{lr}
      Domain & Benchmarks \\\hline
      Math expressions & 29 \\
      Math algorithms & 4 \\
      Physics & 3 \\
      Control & 5 \\
    \end{tabular}
    \caption{Domains from which the \name benchmarks are taken. The
      mathematical expressions are the smallest, and are largely
      drawn from \textit{Numerical Methods for Scientists and
        Engineers}~\cite{hamming-1987}.}
  \end{subtable}
\end{table}

The \name suite attempts balance on multiple dimensions.
Beyond drawing from a range of domains with varying features,
  \name also balances the need for simple benchmarks
  to prove algorithms on
  with larger, practical examples.
To achieve this balance,
  a large number of basic expressions,
  coupled with a smaller number of large test cases.
Among the larger test cases a degree of diversity is also available,
  with programs ranging
  from 2 to 13 variables mutated in the loop body.

\end{document}
