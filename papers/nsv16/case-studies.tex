\documentclass[main.tex]{subfiles}
\begin{document}

\section{Benchmark Suite and Examples}
\label{sec:case-studies}

The \name suite currently includes \nbenches benchmarks
  sourced from recent papers on automatic floating-point
  verification and accuracy improvement.
This section first summarizes these benchmarks
  and then details how representative examples
  were translated to \name from the input formats
  of various tools in the literature.

%% Moreover, it can be used to represent the input
%%   to a variety of floating point tools
%%   which measure error in a variety of ways.

%% \subsection{Benchmark Suite}
%%
%% The \name suite includes \nbenches benchmarks
%%   sourced from recent papers in floating point verification,
%%   optimization, and accuracy improvement.
%% Of the benchmarks,

The current \name suite contains
  examples from a variety of domains, including
  28 from the Herbie test suite~\cite{pavel15},
  9 from the Salsa test suite~\cite{fmics15},
  7 from the Rosa test suite~\cite{DarulovaK14},
  and one example from the FPTaylor test suite~\cite{fptaylor-fm15}.
These examples range from simple test programs for
  early tool development up to
  large examples from industrial control applications for
  evaluating more mature tools.
The larger examples are more challenging,
  including loops that mutate anywhere
  from 2 to 13 variables in the loop body.
As shown in Table~\ref{tbl:domains} and Table~\ref{tbl:features},
  these programs exercise the full range of functionality
  available in \name,
  and span a variety application areas,
  from control software to mathematical libraries.

\begin{table}[hbtp]
  \begin{minipage}[t]{.49\textwidth}
    \begin{tabular*}{\columnwidth}{l@{\extracolsep{\stretch{1}}}r}
      Feature & Benchmarks \\\hline
      Basic Arithmetic & 44 \\
      Exponentials & 13 \\
      Trigonometric & 10 \\
      Comparison & 12 \\
      Loops & 12 \\
      Conditionals & 3
    \end{tabular*}
    \vspace{0.1in}
    \caption{Functions and language features used in the \name
      benchmarks. Benchmarks contain a variety of features, and many
      benchmarks incorporate several. Exponential functions include
      logarithms, the exponential function, and the power function.}
    \label{tbl:domains}
  \end{minipage}
  \hfill
  \begin{minipage}[t]{.49\textwidth}
    \begin{tabular*}{\columnwidth}{l@{\extracolsep{\stretch{1}}}r}
      Domain & Benchmarks \\\hline
      General expressions & 31 \\
      Math algorithms & 6 \\
      Embedded systems & 4 \\
      Scientific computing & 3
    \end{tabular*}
    \vspace{0.25in}
    \caption{Domains which the \name benchmarks are drawn from. Most
      are general mathematical operations, useful in a variety of
      domains. The general expressions are the smallest, and are drawn
      from \textit{Numerical Methods for Scientists and
        Engineers}~\cite{hamming-1987} and Rosa~\cite{DarulovaK14}.}
    \label{tbl:features}
  \end{minipage}
\end{table}

%% The \name suite attempts balance on multiple dimensions.
%% Beyond drawing from a range of domains with varying features,
%%   \name also balances the need for simple benchmarks
%%   to prove algorithms on
%%   with larger, practical examples.
%% To achieve this balance,
%%   a large number of basic expressions,
%%   coupled with a smaller number of large test cases.
%% Among the larger test cases a degree of diversity is also available,
%%   with programs ranging
%%   from 2 to 13 variables mutated in the loop body.

\subsection{FPTaylor}


FPTaylor~\cite{fptaylor-fm15} uses series expansions and interval
arithmetic to compute sound error bounds. The authors gave the following
simple program as an example input for their tool:

\begin{code}
1: Variables
2:   float64 x in [1.001, 2.0],
3:   float64 y in [1.001, 2.0];
4: Definitions
5:   t rnd64= x * y;
6: Expressions
7:   r rnd64= (t-1)/(t*t-1);
\end{code}

This program is representative of the code necessary to correct sensor data
in control software, where the output of the sensor is known to be between
1.001 and 2.0. We manually translated this program to \core, yielding:

\begin{code}
(FPCore (x y)
  :name "FPTaylor example"
  :cite (solovyev-et-al-2015)
  :type binary64
  :pre (and (and (< 1.001 x) (< x 2.0)) (and (< 1.001 y) (< y 2.0)))
  (let ([t (* x y)])
    (/ (- t 1.0)  (- (* t t) 1.0))))
\end{code}

The benchmark takes inputs \C{x} and \C{y}
  and uses a \C{let} statement to represent the intermediate variables
  from the FPTaylor example.
\core faithfully preserves important program constructs,
  such as variable binding and operation ordering,
  making the translation a simple matter.
The benchmark additionally specifies a name
  and cites its source using a key to a standard \BibTeX{} file.
Since the original program uses 64-bit floating-point numbers,
  the type \C{binary64} is specified in the benchmark.
The constraints on input variables are translated to
  a single predicate under the \C{:pre} property.

%% In this case, the preconditions come from
%%   the domain the benchmark is drawn from.

\subsection{Rosa}

Rosa~\cite{DarulovaK14} soundly verifies error bounds of floating-point
programs, including looping control flow through recursive function calls.
Several benchmarks in its repository demonstrate this capability, including
one that uses Newton's method on a series representation of the sine
function.  In Rosa's input language the original benchmark is represented
as:

\begin{code}
def newton(x: Real, k: LoopCounter): Real = {
  require(-1.0 < x && x < 1.0)
  if (k < 10) {
    newton(x - (x - (x**3)/6.0 + (x**5)/120.0 + (x**7)/5040.0) / 
      (1.0 - (x*x)/2.0 + (x**4)/24.0 + (x**6)/720.0), k++)
  } else {
    x
  }
} ensuring(res => -1.0 < res && res < 1.0)
\end{code}

The \C{require} clause denotes input preconditions, and the
\C{ensures} clause provides the error bound to be verified. We manually
translated this program to \core, yielding:

\begin{code}
(FPCore (x0)
  :name "Rosa Example"
  :cite (darulova-kuncak-2014)
  :pre (< (abs x0) 1)
  :rosa-post (< (abs res) 1)
  (while (< i 10)
    ([i 0 (+ i 1)]
     [x x0
      (let ([f (+ (+ (- x (/ (pow x 3) 6))
                     (/ (pow x 5) 120)) (/ (pow x 7) 5040))]
            [df (+ (+ (- 1.0 (/ (* x x) 2)) 
                      (/ (pow x 4) 24)) (/ (pow x 6) 720))])
        (- x (/ f df)))])
    x))
\end{code}

Like the FPTaylor benchmark, this benchmark includes a name,
  citation for its source, and precondition on inputs.
For completeness, we've also included the \C{requires} annotation,
  denoted \C{:rosa-post},
  demonstrating the ability to add tool specific annotations
  by prefixing with the tool name.
Note how the \C{while} construct from \core
  can be used to represent the tail-recursive loop
  from the Rosa original benchmark.

\subsection{Herbie}

Herbie~\cite{pavel15} heuristically improves the accuracy of straight-line
floating-point expressions.  The authors demonstrate the improvements
Herbie can produce using the quadratic formula for computing the roots of a
second degree polynomial.  It has uses from calculating trajectories to
solving matrix equations.  In mathematical notation, the quadratic formula
is given by:%
\footnote{We use the negative variant here, as in the Herbie paper;
  the positive variant is analogous.}

\begin{equation*}
  \frac{(- b) - \sqrt{b^2 - 4ac}}{2a}
\end{equation*}

\noindent Herbie produces the following more-accurate variant:

\newcommand{\K}[1]{\mathbf{#1}\:}

\begin{equation*}
\begin{cases}
  \frac{4ac}{-b + \sqrt{b^2 - 4ac}}/2a & \K{if} b < 0 \\[9pt]
  \left(-b - \sqrt{b^2 - 4ac}\right)\frac1{2a} & \K{if} 0 \le b \le 10^{127} \\[5pt]
  -\frac{b}{a} + \frac{c}{b} & \K{if} 10^{127} < b
\end{cases}
\end{equation*}

\noindent In \core format, the original formula is represented as:

\begin{code}
(FPCore (a b c)
  :name "NMSE p42, positive"
  :cite (hamming-1987)
  :pre (and (>= (sqr b) (* 4 (* a c))) (!= a 0))
  (/ (+ (- b) (sqrt (- (sqr b) (* 4 (* a c))))) (* 2 a)))
\end{code}

\noindent and the improved version is represented as:

\begin{code}
(FPCore (a b c)
  :name "NMSE p42, positive"
  :cite (hamming-1987)
  :pre (and (>= (sqr b) (* 4 (* a c))) (!= a 0))
  (if (< b 0)
      (/ (/ (* 4 (* a c)) (+ (- b) (sqrt (- (sqr b) (* 4 (* a c)))))) 
         (* 2 a))
      (if (< b 10e127)
          (* (- (- b) (- (sqr b) (* 4 (* a c)))) (/ 1 (* 2 a)))
          (+ (- (/ b a)) (/ c b)))))
\end{code}

Like the Rosa and FPTaylor examples,
  this benchmark gives a name, citation, and precondition.
This benchmark uses \core's ability
  to write arbitrary boolean expressions as preconditions,
  restricting the value under the square root to be non-negative
  and requiring the denominator be non-zero.
Unlike the FPTaylor and Rosa examples,
  this precondition arises from mathematical considerations,
  not domain knowledge.
The \core version of Herbie's output
  furthermore uses \C{if} constructs to evaluate
  different expressions for different inputs,
  which improves accuracy.
Herbie could add additional metadata to the output,
  such as its internal estimate of accuracy
  or the number of expressions considered during search,
  by using prefixed keys like \C{:herbie-accuracy-estimate}.

\subsection{Salsa}

Salsa~\cite{fmics15} is a tool for soundly improving the worst-case
accuracy of programs.  The authors evaluate Salsa on a suite of control and
numerical algorithms, including the widely used PID controller algorithm.
This algorithm is used in aeronautic and avionic systems for which
correctness is critical. In C, the benchmark is written as:

\begin{code}
volatile double p, i, t, d, dt, invdt, m, e, eold, r; 
int pid(double m0, double kp, double ki, double kd, double c){
  t     = 0.0;
  invdt = 5.0;
  dt    = 0.2;
  m     = m0 ;
  eold  = 0.0;
  i     = 0.0;
  while (t < 100.0) {
       e = c - m;
       p = kp * e;
       i = i + ki * dt * e;
       d = kd * invdt * (e - eold);
       r = p + i + d;
       m = m + 0.01 * r; /* computing measure: the plant */
       eold = e;
       t = t + dt;
  }
  return m;
}
\end{code}

To ease the conversion of this code from C to \core,
  this program was first manually translated
  to the following \surface program:

\begin{code}
(FPImp (m kp ki kd)
 :name "PID"
 :description "Keep a measure at its setpoint using a PID controller."
 :cite (damouche-martel-chapoutot-nsv14)
 :type binary64
 :pre (and (and (< -10.0 m) (< m 10.0)) (and (< -10.0 c) (< c 10.0)))
 [= t 0.0]
 [= dt 0.2]
 [= invdt (/ 1 dt)]
 [= c 0.0]
 [= eold 0.0]
 [= i 0.0]
 (while (< t 100.0)
   [= e (- c m)]
   [= p (* kp e)]
   [= i (+ i (* (* ki dt) e))]
   [= d (* (* kd invdt) (- e eold))]
   [= r (+ (+ p i) d)]
   [= m (+ m (* 0.01 r))]
   [= eold e]
   [= t (+ t dt)])
 (output m))
\end{code}

The \surface program was then automatically compiled,
  using the compiler tool in \name,
  to the following \core benchmark:

\begin{code}
(FPCore (m0 kp ki kd c)
  :name "PID"
  :description "Keep a measure at its setpoint using a PID controller."
  :cite (damouche-martel-chapoutot-nsv14)
  :type binary64
  :pre (and (and (< -10.0 m0) (< m0 10.0)) (and (< -10.0 c) (< c 10.0)))
  (while (< t 100.0)
   ([i 0.0 (+ i (* (* ki 0.2) (- c m)))]
    [m m0
     (let ([p (* kp (- c m))]
           [d (* (* kd (/ 1 0.2)) (- (- c m) eold))])
     (+ m (* 0.01 (+ (+ p (+ i (* (* ki 0.2) (- c m)))) d))))]
    [eold 0.0 (- c m)]
    [t 0.0 (+ t 0.2)])
   m))
\end{code}

Beyond the metadata used in the FPTaylor, Rosa, and Herbie examples,
  this benchmark includes a \C{:description} tag
  to describe for readers what the benchmark program computes.
These descriptions contain information about the distribution of inputs,
  the situation in which the benchmark is used,
  or any other information which might be useful to tool writers.
The \name suite features descriptions for its most complex benchmarks.

\end{document}
