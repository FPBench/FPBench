\documentclass[main.tex]{subfiles}
\begin{document}

\section{Examples}

FPBench supports benchmarks from a variety of domains, with a variety
of characteristics. Here, we'll highlight a few representative
examples supported by FPBench, and describe their characteristics
briefly.

\subsection{Quadratic Formula}

The quadratic formula is a common formula used to compute the roots of
a degree two polynomial. It appears in \textit{Numerical Methods for
  Scientists and Engineers}, by Richard Hamming, in a section on
rearranging computations for numerical accuracy. It has uses in a
variety of applications, from calculating trajectories, to solving
matrix equations. In mathematical notation, the quadratic formula is
written as:

\begin{equation}
  \frac{(- b) \pm \sqrt{b^2 - 4ac}}{2a}
\end{equation}

In FPBench format, the positive variant of quadratic formula is
represented as:

\begin{verbatim}
(lambda (a b c)
  :name "NMSE p42, positive"
  :cite (hamming-1987)
  :pre (and (>= (sqr b) (* 4 (* a c))) (not (= a 0)))
  (/ (+ (- b) (sqrt (- (sqr b) (* 4 (* a c))))) (* 2 a)))
\end{verbatim}

The negative variant is represented similarly, so we'll only describe
the positive variant here.

The benchmark takes three inputs, denoted ``a'', ``b'', and ``c''
above. It is named ``NMSE p42, positive'' after the book in which it
appears and its exercise number, as well as which branch it is. It
cites its source through a shorthand, ``hamming-1987'', which
corresponds to a .bib file included with the benchmark suite.

The benchmark denotes a range of acceptable inputs, through
preconditions. For the quadratic formula, since it is ill defined to
take the square root of a negative number, we restrict the input to
the square root to be non-negative. Additionally, it is ill defined to
divide by zero, so we restrict the denominator of the division to be
non-zero, by restricting the input ``a'' to be non-zero.

Finally, the last field of the benchmark is the program
expression. This is written in FPCore syntax, a s-expression based
syntax described above.

\end{document}
