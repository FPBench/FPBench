\documentclass[main.tex]{subfiles}
\begin{document}

\section{Conlusion}
\label{sec:conclusion}

The initial work on FPBench presented in this article provides a basic
foundation for comparing and evaluating floating point analysis and
optimization tools.  Already the FPBench format can serve as a
\textit{lingua franca} between such tools and enable them to
cooperatively analyze and improve floating point code.

Of course, there is still much work to do in order to make FPBench
sufficiently expressive for use by the broad range of interests represented
in the floating point community.  In the near term, we will add additional
metrics for accuracy and performance to the set of evaluators provided by
the FPBench tooling and begin developing a standard set of benchmarks
around the various measures.  We will also expand the set of languages with
direct support for compilation to and from the FPBench format.  Longer
term, we intend to investigate support for mixed precision benchmark,
including supporting fixed point computations.

We hope that FPBench can help foster the already strong sense of community
and collaboration around floating point research.  Toward that end, we
encourage any interested readers to get involved with development of
FPBench by signing up for the mailing list and checking out the project
website at \url{http://fpbench.org/}.

\end{document}
