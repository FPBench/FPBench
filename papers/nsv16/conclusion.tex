\documentclass[main.tex]{subfiles}
\begin{document}

\section{Conclusion}
\label{sec:conclusion}

The initial work on \name provides a
foundation for evaluating and comparing floating-point analysis and
optimization tools.  Already the \name format can serve as a
common language, allowing these tools to
cooperatively analyze and improve floating-point code.

Though \name supports the composition
  of the floating-point tools that exist today,
  there is still much work to do to support
  the floating-point research community as it grows.
\name must be sufficiently expressive for the broad range of applications
  that represent the future of the community.
In the near term, we will add additional
metrics for accuracy and performance to the set of evaluators provided by
the \name tooling and begin developing a standard set of benchmarks
around the various measures. We will also expand the set of languages with
direct support for compilation to and from the \name format.
As more tools grow support for \name,
  we will provide automated comparisons of different floating-point tools.
Longer term, we intend to support for mixed-precision benchmarks,
  fixed-point computations, and additional data structures
  such as vectors, records, and matrices.

We hope that \name encourages the already strong sense of community
and collaboration around floating-point research.  Toward that end, we
encourage any interested readers (and tool writers)
to get involved with development of \name
by signing up for the mailing list and checking out the project
website: 

\begin{center}
\large\url{http://fpbench.org/}
\end{center}

\end{document}
