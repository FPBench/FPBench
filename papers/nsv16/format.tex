\documentclass[main.tex]{subfiles}
\begin{document}

\section{Benchmark Format}
\label{sec:format}

The \name format must be easy to parse and have simple, easily-modeled
semantics that allow the details of floating point arithmetic to come
to the fore. These goals are fulfilled by the \core format, a
simplified expression-based language for floating point operations
inspired by the \smtlib format used in the SMT competition.

The \core syntax is documented in Figure~\ref{fig:core}.

\begin{figure}
\begin{grammar}
<benchmark> ::= ( `lambda' ( <argument>* ) <property>* <expression> )

<argument> ::= <identifier>

<property> ::= <propname> ` ' <propval>

<expression> ::= <number> | <constant> | <identifier>
\alt ( <operation> <expression>* )
\alt ( `if' <expression> <expression> <expression> )
\alt ( `while' <expression> ( <loopvar>* ) <expression> )

<loopvar> ::= [ <identifier> <expression> <expression> ]

<propval> ::= <expression> | <string> | ( <identifier>+ ) | <identifier>

\end{grammar}

\end{figure}

\end{document}
