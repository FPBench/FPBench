\documentclass[main.tex]{subfiles}
\begin{document}

\section{Benchmark Format}
\label{sec:format}

The \name format must be easy to parse and have simple, easily-modeled
semantics that allow the details of floating point arithmetic to come
to the fore. These goals are fulfilled by the \core format, a
simplified expression-based language for floating point operations
inspired by the \smtlib format used in the SMT competition.

\begin{figure}
\begin{grammar}
<benchmark> ::= ( `lambda' ( <identifier>* ) <property>* <expression> )

<property> ::= <propname> ` ' <propval>

<expression> ::= <number> | <constant> | <identifier>
\alt ( <operation> <expression>* )
\alt ( `if' <expression> <expression> <expression> )
\alt ( `while' <expression> ( <loopvar>* ) <expression> )

<loopvar> ::= [ <identifier> <expression> <expression> ]

<propval> ::= <expression> | <string> | ( <identifier>* ) | <identifier>

<constant> ::= `E' | `LOG2E' | `LOG10E' | `LN2' | `LN10' | `PI' | $\dotsb$

<operation> ::= `+' | `*' | `<' | $\dotsb$ | `abs' | `acos' | `and' | `asin' | $\dotsb$

\end{grammar}
\caption{The grammar of the \core language. Since the language is an
  S-expression format, parentheses and braces are literal in this
  grammar. See Figure~\ref{fig:supported} for supported constants and
  operations.}
\label{fig:core}
\end{figure}

The \core syntax is documented in Figure~\ref{fig:core}. It allows
ordinary function calls, along with \C{if} statements and \C{while}
loops. The semantics are standard functional semantics with the
following rule for while loops:

\begin{mathpar}
\inferrule{
  H[x_i \mapsto x_i^0]: c \evalto \top \\
  H[x_i \mapsto x_i^0]: e_i \evalto x_i'
}{
  H: (\C{while}\:([x_i\:x_i^0\:e_i]_i)\:y)
  \to
  H: (\C{while}\:([x_i\:x_i'\:e_i]_i)\:y)
}

\inferrule{
  H[x_i \mapsto x_i^0] : c \evalto \bot \\
  H[x_i \mapsto x_i^0] : y \evalto w
}{
  H: (\C{while}\:([x_i\:x_i^0\:e_i]_i)\:y)
  \evalto
  w
}
\end{mathpar}

\end{document}
