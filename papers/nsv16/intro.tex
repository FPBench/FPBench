\documentclass[main.tex]{subfiles}
\begin{document}

\section{Introduction}
\label{sec:intro}


In a world where trusting software systems is increasingly important, automatic program verification tools are mandatory to provide certified results. In this context floating-point computations are an important issue because of the rounding errors done
on each operation. 
Many validating methods concerning floating-point arithmetic have been introduced.  
Among them, we may cite: 
Fluctuat~\cite{GMP06,Gou13} measures by static analysis an over-approximation of the error due to the use of floating-point numbers instead of reals while executing a program in \texttt{C} language.  
It also, helps the user to debug its code by detecting  the responsible operations of the most significant precision loss. 
This approach is used by many industries.    
The Salsa tool~\cite{fmics15} improves the numerical accuracy of programs by automatic transformation. 
It minimizes the value of the errors arising during computations using error bounds obtained by abstract interpretation.
Based on a search to improve the numerical accuracy of an arithmetic expression, Herbie~\cite{pavel} (Zachary and al.) estimates and localizes the roundoff errors of an expression involving sampled points, applies a set of rules in order to improve the accuracy of the expression, takes series expansions and combines these improvements for the different inputs.  
Another approach introduced by~\cite{DarulovaK14} and implemented in the Rosa tool intends to combine an exact SMT solver based on reals with an approximate and sound affine and interval arithmetic computation. 
Its use required to set a desired postcondition and involves the uncertainties as well as the desired target precision. 
The compiler verifies that the desired precision could be soundly obtained in finite-precision implementation while all the uncertainties and their propagation are included. Finally, the FP-Taylor tool uses Taylor series developments to narrow
the error computed by the interval arithmetic. This competes with the affine arithmetic domain used by Fluctuat.





There is an explosion of floating point tools, including
FPTaylor~\cite{fptaylor-fm15}.

In summary, this paper contributes:
%
\begin{enumerate}
%
  \item An expressive format for floating point point benchmarks.
%
  \item A set of measures for evaluating performance of various tools on
  these benchmarks.
%
  \item Examples taken from existing tools in the literature an shown to be
  expressible in FPBench.
%
\end{enumerate}

\end{document}
