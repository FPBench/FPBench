\documentclass[main.tex]{subfiles}
\begin{document}

\section{Introduction}
\label{sec:intro}

In a world where trusting digital systems is increasingly important,
automatic program verification tools are mandatory to provide
certified results.  In this context floating-point computations are an
important issue because of the rounding errors done on each operation.
Indeed, floating-point arithmetic is not intuitive and very sensitive
to the roundoff errors.  To ensure the validity of the results
obtained with this arithmetic, several validating methods have been
introduced.  Among them, we may cite: Fluctuat~\cite{Goubault13,GMP06}
measures by static analysis an over-approximation of the error due to
the use of floating-point numbers instead of reals while executing a
program written in the \texttt{C} language.  It also, helps the user
to debug its code by detecting the responsible operations of the most
significant precision loss.  This approach is used by several
industries.  Salsa~\cite{fmics15} improves the numerical accuracy of
programs by automatic transformation.  It minimizes the value of the
errors arising during computations using error bounds obtained by
abstract interpretation.  Based on a search to improve the numerical
accuracy of an arithmetic expression, Herbie~\cite{pavel15} (Zachary
and al.) estimates and localizes the roundoff errors of an expression
using sampled points, applies a set of rules in order to improve the
accuracy of the expression and combines these improvements for the
different inputs.  Another approach introduced by~\cite{DarulovaK14}
and implemented in Rosa intends to combine an exact SMT solver based
on reals with an approximate and sound affine and interval arithmetic
computation.  Its use requires to set a desired post-condition and
involves the uncertainties as well as the desired target precision.
The compiler verifies that the desired precision can be soundly
obtained in finite-precision implementation while all the
uncertainties and their propagation are included.  Finally,
FP-Taylor~\cite{fptaylor-fm15} uses Taylor series developments to
narrow the error computed by the interval arithmetic.  This competes
with the affine arithmetic domain used by Fluctuat.


While there is a larger and larger number of tools dedicated to
improve the numerical accuracy of codes and to bound the errors
arising in floating-point computations, it becomes more and more
difficult to compare them on the same programs because of the absence
of a standard format and relevant suite for benchmarks and the fact
tools do not use the same measures of the accuracy.  Hence, the
present work is motivated by the absence of a global approach that
provides us the opportunity to compare the different tools previously
mentioned.  In addition, having a common scientific methodology is
very important to show the improvements of each tool.


The main contributions of this article are the following:
\begin{enumerate}[label=(\roman*)]
\item We define benchmarks with lots of examples coming from different
  domains and translated to the \texttt{FPBench} language.
\item We propose several ways to measure the errors. 
\item We indicate what compute the main tools. 
\item We give then experimental results observed.
\end{enumerate}


This article is outlined as following. In Section~\ref{sec:format}, we
recall some basic notions on term representing as well as the
benchmark format. In Section~\ref{sec:measure} we introduce the
different kinds of measures of the error. Section~\ref{sec:tools}
describes a set of tools released with FPBench to make creating and
working with benchmarks easier. We discuss in
Section~\ref{sec:case-studies} our existing benchmark suite,
highlighting a few representative case studies. Finally, in
Section~\ref{sec:conclusion}, conclusions are drawn and further work
is suggested.




%There is an explosion of floating point tools, including
%FPTaylor~\cite{fptaylor-fm15}.

%In summary, this paper contributes:
%
%\begin{enumerate}
%
%  \item An expressive format for floating point point benchmarks.
%
%  \item A set of measures for evaluating performance of various tools on  these benchmarks.
%
% \item Examples taken from existing tools in the literature an shown to be   expressible in FPBench.
%
%\end{enumerate}

\end{document}
