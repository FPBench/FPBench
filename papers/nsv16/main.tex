\documentclass{llncs}
\usepackage{amsmath,amssymb,doc}
\usepackage{subfiles,graphicx}
\usepackage{xspace}
\usepackage{syntax,mathpartir,listings}
\usepackage[usenames,dvipsnames]{color}
\usepackage{enumitem,hyperref}

\newcommand{\name}{FPBench\xspace}
\newcommand{\core}{FPCore\xspace}
\newcommand{\surface}{FPImp\xspace}
\newcommand{\nbenches}{44\xspace}
\newcommand{\smtlib}{\textsc{smt-lib}\xspace}
\newcommand{\C}[1]{\texttt{#1}\xspace}
\newcommand{\evalto}{\Downarrow}
\newcommand{\err}[1]{\varepsilon_{\text{#1}}}
\lstnewenvironment{code}{\lstset{basicstyle=\scriptsize}}{}
\newcommand{\todo}[1]{{\color{blue}\textit{[TODO: #1]}}}

\begin{document}

\title{Toward a Standard Benchmark Format and Suite for Floating Point Analysis}
\author{
  Nasrine Damouche\textsuperscript{1} \hspace{0.1in}
  Matthieu Martel\textsuperscript{1} \hspace{0.1in}
  Pavel Panchekha\textsuperscript{2} \\
  Chen Qiu\textsuperscript{2} \hspace{0.1in}
  Alexander Sanchez-Stern\textsuperscript{2} \hspace{0.1in}
  Zachary Tatlock\textsuperscript{2}
}
\institute{
  Université de Perpignan Via Domitia\textsuperscript{1} \hspace{0.1in}
  University of Washington\textsuperscript{2}
}
\maketitle

%% \subfile{outline}

\subfile{abstract}
\subfile{intro}
\subfile{format}
\subfile{measure}
\subfile{tools}
\subfile{case-studies}
\subfile{conclusion}

\bibliographystyle{abbrv}
\bibliography{fpbench}

\end{document}
