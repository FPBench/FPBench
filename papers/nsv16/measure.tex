\documentclass[main.tex]{subfiles}
\begin{document}

\section{Accuracy Measurements}
\label{sec:measure}

Floating point accuracy tools can measure their performance and the
performance of the programs they operate on in a number of ways. In
order to fully support such tools, FPBench supports many different
ways of measuring program error and its improvement. Here, we'll
describe several axis on which choices about measuring floating point
error can be made, and the various options on each.

To even begin to write a tool which measures and manipulates floating
point error, one must start by asking several questions:

\begin{itemize}
\item How do you measure how ``far apart''  two floating point
  numbers are?
\item Do you care about the difference in output for a correct input
  (forward error), or the difference in input for a correct output
  (backward error)?
\item Do you measure error statically (through program analysis), or
  dynamically (through sampling)?
\item How do you aggregate error across an entire input space?
\item For tools that manipulate floating point programs: how do you
  measure error improvement?
\end{itemize}

Each of these questions have multiple valid answers, and we'll address
the common ones here.

\begin{description}
\item[How do you measure the how ``far apart'' two floating point numbers are?]

  There are several ways to measure the difference between two
  floating point numbers, $x$ and $x'$. The simplest method is
  absolute error:
  \begin{equation}
    \varepsilon_{abs}(x, x') = \left|x - x'\right|
  \end{equation}
  This method is simple to compute and understand. But many
  mathematicians prefer to measure how big the error is relative to
  the quantity being measured. From this notion, we get relative
  error:
  \begin{equation}
    \varepsilon_{rel}(x, x') = \left|\frac{x - x'}{x}\right|
  \end{equation}
  Relative error scales with the magnitude of the quantity being
  measured, and thus makes sense for measuring both large and small
  numbers, much like the floating point format itself. To tie our
  notion of error even more closely to the floating point format, we
  can count the floating point values between $x$ and $x'$.
  \[
  \varepsilon_{ulps}(x, x') = |\{ y \in \mathbb{F} : \min(x, x') \le y \le \max(x, x')
  \}|~\footnote{We are using $|S|$ to denote the number of elements in
    a set $S$}
  \]
  The floating point numbers are distributed roughly exponentially, so
  this measure of error scales in a similar manner to relative
  error. Since it's more faithful to the underlying representation,
  this measure, commonly referred to as ULPs or Units in the Last
  Place, is often a smoother measure of error as it appears in
  floating point error. Some tools even take the $log$ of this measure to
  get roughly the number of low-order bits which are incorrect in the
  answer.
\end{description}

%% There exists many ways of measuring the accuracy of a computation and
%% it is mandatory to be very precise on which methodology is adopted in
%% order to compare techniques or tools. Even if no existing tool
%% implements all the measures enumerated hereafter, our goal is to
%% propose a taxonomy of the main accuracy measurement methods. First of
%% all, we must distinguish between the following alternatives:

%% \begin{itemize}
%% \item The kind of error considered in  benchmarks, i.e. \textit{which error} is used. The kind of error used in a benchmark may be the absolute or relative error \cite{Gol91}. While tools use in general forward errors, 
%% backward errors are usually more relevant but also more difficult to compute \cite{Chat}.
%% \item The object of the measure, i.e. \textit{what is measured}. For tools estimating the accuracy of a computation, the object of the measure is typically
%% the worst case or the average error possibly arising during a computation for all the considered inputs.
%% Other measures may be additionnaly considered for accuracy optimizing tools like the average improvement,
%% the worst case improvement or the worst anti-improvement.
%% \item The technique used to achieve the measure, i.e. \textit{how it is measured}. The most employed techniques rely
%% on over-approximations of the errors obtained, for example, using interval computations or statistical approximations
%% obtained by repeating the measurement on sample inputs the appropriate number of times.
%% \end{itemize}
 
%% We explicit hereafter the elements of measures enumerated previously. First of all, concerning the kind of
%% error, let $x\in X$  be some input, let $f\ :\ X\rightarrow Y$ be a function assumed to be exact and 
%% let $\hat{f}\ :\ X\rightarrow Y$ be another function approximating $f$ (note that
%% the domain and co-domain of $\hat{f}$ may be subsets of $X$ and $Y$, for example floating-point numbers are a subset of the real numbers).
%% The absolute error $\varepsilon_{abs}$ and relative error $\varepsilon_{rel}$  
%% are defined by
%% \begin{equation}
%% \varepsilon_{abs}(x) = | f(x) - \hat{f}(x)|\quad\text{and}\quad \varepsilon_{rel}(x)= \left| \frac{f(x)-\hat{f}(x)}{f(x)}\right|\enspace .
%% \end{equation}
The error $\varepsilon_{abs}(x)$ corresponds to the forward error. The backward error $\varepsilon_{back}(x)$ is defined as the distance
to the closest input $x'$ such that $f(x')=\hat{f}(x)$ or, in other word, the distance to the closest input
for which $\hat{f}(x)$ is the exact result.
\begin{equation}
\varepsilon_{back} = \min\ \big\{ x'\in X\ :\ f(x')=\hat{f}(x)\big\}\enspace .
\end{equation}
To our knowledge, all the existing tools measuring automatically errors on pieces of codes compute forward errors and,
in the following, we only consider this kind of error.

The object of the measure refers to what quantity is measured.  Let $\varepsilon(x)$
stand for either $\varepsilon_{abs}(x)$ or $\varepsilon_{rel}(x)$ depending on the kind of error we use. 
We may distinguish the following cases.
\begin{description}
\item[Worst case error] The worst case error $\varepsilon_{wst}$ is the error measured in the worst case for all the admissible inputs of the program. 
\begin{equation}
\varepsilon_{wst} = \max\ \big\{ \varepsilon(x)\ :\ x\in X \big\} \enspace .
\end{equation}
\item[Average error] The average error $\varepsilon_{avg}$ is the mean error measured for the considered inputs. Let Card$(X)$ denote
the cardinal of the set $X$.
 \begin{equation}
\varepsilon_{avg} = \frac{1}{\text{Card(X)}} \cdot \sum_{x\in X} \varepsilon(x) \enspace .
\end{equation}
\item[Worst anti-improvement] In general, accuracy optimizing tools cannot improve the accuracy of some computation for
all the acceptable inputs (apart by generating as many specialized versions of the code as needed). Consequently,
beside the worst case and average errors, other measures are relevant. The worst anti-improvement $\iota_{anti}$
measures the additional error added by the transformation in the worst case. Let $\hat{f}_{opt}$ be an accuracy optimized
version of $\hat{f}$. It is admitted that $\varepsilon_{wst}(\hat{f}_{opt})< \varepsilon_{wst}(\hat{f})$
or $\varepsilon_{avg}(\hat{f}_{opt})< \varepsilon_{avg}(\hat{f})$. Nevertheless, $\hat{f}_{opt}$ may be
less accurate than $\hat{f}$ on some input.
\begin{equation}
\iota_{anti}(\hat{f},\hat{f}_{opt}) = \max \left\{ \big| |f(x)-\hat{f}(x)| - |f(x)-\hat{f}_{opt}(x)|   
\big|\ :\ x\in X\right\}
\end{equation}
\item[Worst case and average improvement] The worst case improvement $\iota_{wst}$ is the improvement of error in the worst case. The average improvement $\iota_{avg}$ is the mean improvement of error measured 
for the considered inputs.  They are defined by
 \begin{equation}
\iota_{wst} = \varepsilon_{wst}(\hat{f}) - \varepsilon_{wst}(\hat{f}_{opt})\enspace,
\end{equation}
 \begin{equation}
\iota_{avg} = \frac{1}{\text{Card(X)}} \cdot \sum_{x\in X}  \big| |f(x)-\hat{f}(x)| - |f(x)-\hat{f}_{opt}(x)|  \big|\enspace .
\end{equation}
\end{description}


Note that other measures could be considered. For example, measures performing a trade-off between accuracy of the
results and performances of the code would be of great interest. Other criteria, like the impact of the transformation
on the numerical stability of the code could be taken into account. These measures could rely on the value
of the derivatives of the code obtained by automatic differentiation \cite{} or by the computation of Lyapunov exponents \cite{}.


\textbf{\large Work in progress below}

Independently of the object measured,
there exists two families of techniques to compute the errors based on static or dynamic analysis of the codes.
Dynamic techniques take a subset $X' \subseteq X$ of the inputs, in general by chosing random elements of $X$,
and compute $\varepsilon_{wst}$,  $\varepsilon_{avg}$, $\iota_{anti}$, $\iota_{avg}$, $\ldots$ using $X'$
instead of $X$. While dynamic techniques avoid to over-approximate the errors as it may happens when static 
techniques are used, the errors may also be under-approximated which may be an issue in some applicative
contexts such as critical embedded systems. Another difficulty of dynamic methods is the execution time of the
multiple runs, specially for programs with multiple inputs. In this case, the way random inputs are generated
must be specified. For example, for a program taking two inputs $x$ and $y$, one may take $n$ sample values for $x$ and
$n$ other values for $y$ and execute the $n\times n$ tests of the cartesian product or, alternatively, one may generate
a suite of $n$ pairs of data $(x,y)$ and execute the $n$ tests. Obviously, this impacts the execution time and the scalability of the tool
as well as how we may trust the results. Let us also remark that probabilities may be used in another ways, for example
as in the CESTAC method, implemented in the Cadna tool \cite{} in which each operation is run several times with random rounding modes
(towards plus or minus infinity with a probability of $0.5$).

Static analysis techniques compute an other-approximation of the errors without executing the programs. They rely
on non-relational \cite{} or relational abstract domains \cite{}. Static analysis techniques also use 
acceleration techniques (widenings) to over-approximate the results of programs with loops without executing all the iterations.
While this is another source of other-approximation, it makes it possible to obtain in a short time general results which hold for 
programs independently of the number of iterations.


\scriptsize
\centerline{\begin{tabular}{ccccc}
%{|p{1.5cm}|p{2.5cm}|p{2.5cm}|p{2.5cm}|p{2.5cm}|}
\hline 
         & Object of Meas. & Technique of Meas. & Measured Objects & Optimized Objects \\
\hline 
Fluctuat & Worst-Case        & Static Analysis      & Commands         & -\\
FP-Taylor& Worst-Case        & Static Analysis      & Expressions      & -\\
Herbie   & Average           & Dynamic Analysis     & Expressions      & Expressions \\
Rosa     & Worst-Case        & Static Analysis      & Expressions      & -\\
Salsa    & Worst-Case        & Static Analysis      & Commands         & Commands \\
\hline
\end{tabular}
}
\end{document}
