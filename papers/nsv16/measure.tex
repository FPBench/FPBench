\documentclass[main.tex]{subfiles}
\begin{document}

\section{Accuracy Measurements}
\label{sec:measure}

To begin to compare
  floating-point benchmark results across tools,
  tools must have a common measure of error.
However, given the diversity of tools and error measurements
  that exist in the literature,
  standardizing on a single measure of error
  would not only be difficult but could harm the development
  of some classes of tools.
Instead, \name standardizes the language
  for discussing error measurements
  to facilitate using the \name suite to compare floating-point tools.

\subsection{Measurement Types}

Floating-point error is best analyzed along several axes:
  scaling vs non-scaling difference, forward vs. backward error,
  maximum vs. average error, sound vs. statistical techniques,
  and, for tools which transform programs,
  how to measure improvement.

\paragraph{Scaling vs non-scaling difference ($\varepsilon$)}

There are several ways to measure the error incurred
  by producing the inaccurate value $\hat x$ instead of the correct value $x$.
Two common mathematical notions are the absolute and relative error:
\begin{equation*}
  \varepsilon_{abs}(x, x') = \left|x - \hat x\right|
  \quad \text{and} \quad
  \varepsilon_{rel}(x, x') = \left|\frac{x - \hat x}{x}\right|
\end{equation*}
Relative error scales with the quantity being measured,
  and thus makes sense for measuring both large and small numbers,
  much like the floating-point format itself.
A notion of error more closely tied to the floating-point format
  is the Units in the Last Place (ULPs) of error.%
\footnote{We are using $|S|$ to denote the number of elements in a set $S$}
\[
\varepsilon_{ulps}(x, x') = |\{ y \in \mathbb{F} : \min(x, \hat x) \le y \le \max(x, \hat x)
\}|
\]
The floating-point numbers are distributed roughly exponentially, so
this measure of error scales in a similar manner to relative
error; however, it is better-behaved in the presence of denormal
numbers and infinities.
Some tools use the logarithm of the ULPs to roughly approximate
the number of incorrect low-order bits in $\hat x$.

\paragraph{Forward vs. backward error ($\epsilon$)}

Forward error and backward error are two common measures
  for the error of a mathematical \emph{computation}.
For a correct function $f$ and its approximation $\hat f$,
  forward error measures the difference between outputs for a fixed input,
  while backward error measures the difference between inputs
  for a fixed output.
Formally,\footnote{Where $n$ is the number of arguments.}

\begin{equation*}
  \epsilon_{fwd}(x) = \varepsilon(f(x), \hat{f}(x))
  \quad \text{and} \quad
  \epsilon_{bwd}(x) =
  \min \left\{ \varepsilon(x, x') : x' \in \mathbb{F}^n \land {\hat f}(x') = f(x) \right\}.
\end{equation*}

Backward error is important for evaluating the stability of an algorithm,
  and in scientific applications where multiple sources of error
  (algorithmic error vs. sensor error) must be compared.
However, backward error can be tricky to compute,
  especially for floating-point computations
  where there may not be an input that produces the correct output.
To our knowledge, all existing tools measure forward error.

\paragraph{Average vs. maximum error ($E$)}

Describing the error of a floating-point computation
  means summarizing its behavior across multiple inputs.
Existing tools use either maximum or average error for this task.
Formally,\footnote{Where $N$ is the number of valid inputs, and $n$ is the number of arguments.}

\begin{equation*}
  E_{\text{max}} = \max \left\{\epsilon(x) : x \in \mathbb{F}^n\right\}
  \quad \text{and} \quad
  E_{\text{avg}} = \frac{1}{N} \sum_{x\in \mathbb{F}^n} \epsilon(x) \enspace .
\end{equation*}

Worst-case error tends to be easier to measure soundly,
  while average error tends to be easier to measure statistically.

\paragraph{Sound vs. statistical techniques}

It is intractable to evaluate the error of the program on all valid inputs.
Existing tools use
  either static analyses to soundly overapproximate the error
  or statistical sampling to tightly approximate the error.

Most static techniques are based on interval or affine arithmetic
  to over-approximate floating-point arithmetic,
  often using abstract interpretation.
Abstract interpretation may be either non-relational \cite{Mar05}
  or relational abstract domains \cite{Cha10,GP11,AGW15},
  and may use acceleration techniques (widenings \cite{CC92})
  to over-approximate loops without unrolling them.
While such techniques tend to provide loose over-approximations
of the floating-point error of programs, they are fast and
provide sound error bounds. In some embedded applications,
correctness is critical and unsound techniques will not do.

In domains where correctness is not absolutely critical,
dynamic sampling techniques can provide tighter approximations of
error as it appears in practice. Dynamic error evaluation techniques
tend to be slower than static techniques, since they involve running a
program multiple times, sometimes with varying semantics.

\paragraph{Measuring improvement ($\iota$)}

Tools that transform floating point programs,
  such as those that attempt to improve the accuracy of programs,
  often compare the accuracy of two floating-point programs
  instead of measuring the accuracy of a single program.
Such comparison uses the improvement in worst-case or average error
  between the original $\hat f$ and improved $\hat f'$ implementation
  of the mathematical function $f$:
\begin{align*}
  \iota_{\text{imp}} = E(\hat{f}) - E(\hat{f}')
\end{align*}

One cannot, in general,
  improve the accuracy of a computation
  simultaneously on all inputs.
It is thus often necessary
  to make a computation less accurate on some points
  to make it more accurate overall.
In this case, it may be useful to report the largest unimprovement,
  which measures the cost of improving accuracy:

\begin{equation*}
  \iota_{\text{wrs}}(\hat{f},\hat{f}') = \max \left\{ \epsilon(f(x),\hat{f}'(x)) - \epsilon(f(x),\hat{f}(x))   
\ :\ x\in \mathbb{F}^n\right\}  
\end{equation*}

Other measures,
  such as those describing the trade-off between accuracy and speed,
  are also interesting, but are less commonly used in the literature
  and thus not standardized.
Similarly,
  improvement tools could also estimate their effect on numerical stability
  using automatic differentiation~\cite{GW08} or Lyapunov exponents~\cite{Sri13},
  but we do not know of any such tools.
%  \todo{add citations here.}

\subsection{Existing Tools}

The error measures described can be applied
  to categorize the error measurements used by existing tools.
Table~\ref{tbl:tools} compares Fluctuat~\cite{Goubault13},
  FPTaylor~\cite{fptaylor-fm15}, Herbie~\cite{pavel15},
  Rosa~\cite{DarulovaK14}, and Salsa~\cite{fmics15}.

\begin{table}[htb]
\begin{tabular*}{\columnwidth}{@{\extracolsep{\stretch{1}}}*{6}{l}}
Fluctuat & Absolute        & Forward & Max      & Sound \\
FPTaylor & Absolute        & Forward & Max      & Sound \\
Herbie   & ULPs            & Forward & Average     & Statistical & Improvement \\
Rosa     & Absolute        & Forward & Max      & Sound \\
Salsa    & Absolute        & Forward & Max      & Sound & Improvement
\end{tabular*}
\caption{A comparison of how five floating-point measure error
  across the axes identified in this section.}
\label{tbl:tools}
\end{table}

Fluctuat, FPTaylor, and Rosa all verify error bounds
  on the accuracy of floating-point computations.
Given their need for soundness, it is natural
  that they use sound error analyses and estimate maximum error.
Their use of absolute forward error derives from the difficulty
  of approximating the other forms of error statically.
Herbie and Salsa are tools for improving the accuracy of programs,
  but differ dramatically in their approach.
Salsa uses abstract interpretation to bound maximum absolute error,
  producing a sound overapproximation of the maximum error.
Herbie, on the other hand, uses random sampling to achieve
  a tight statistical approximation of ULP error.
The tight estimates enabled by statistical techniques
  provide additional opportunities for Herbie
  to improve the accuracy of an expression,
  but prevent it from providing sound error bounds.
Unsound tools such as Herbie can be supplemented
  by using a sound verification tool on its output,
  a composition enabled by the common \core format.
Meanwhile, since Fluctuat, FPTaylor, Rosa, and Salsa
  all soundly measure maximum forward absolute error,
  their measurements can be compared to determine which technique is best.

%% There exists many ways of measuring the accuracy of a computation and
%% it is mandatory to be very precise on which methodology is adopted in
%% order to compare techniques or tools. Even if no existing tool
%% implements all the measures enumerated hereafter, our goal is to
%% propose a taxonomy of the main accuracy measurement methods. First of
%% all, we must distinguish between the following alternatives:

%% \begin{itemize}
%% \item The kind of error considered in  benchmarks, i.e. \textit{which error} is used. The kind of error used in a benchmark may be the absolute or relative error \cite{Gol91}. While tools use in general forward errors, 
%% backward errors are usually more relevant but also more difficult to compute \cite{Chat}.
%% \item The object of the measure, i.e. \textit{what is measured}. For tools estimating the accuracy of a computation, the object of the measure is typically
%% the worst case or the average error possibly arising during a computation for all the considered inputs.
%% Other measures may be additionnaly considered for accuracy optimizing tools like the average improvement,
%% the worst case improvement or the worst anti-improvement.
%% \item The technique used to achieve the measure, i.e. \textit{how it is measured}. The most employed techniques rely
%% on over-approximations of the errors obtained, for example, using interval computations or statistical approximations
%% obtained by repeating the measurement on sample inputs the appropriate number of times.
%% \end{itemize}
 
%% We explicit hereafter the elements of measures enumerated previously. First of all, concerning the kind of
%% error, let $x\in X$  be some input, let $f\ :\ X\rightarrow Y$ be a function assumed to be exact and 
%% let $\hat{f}\ :\ X\rightarrow Y$ be another function approximating $f$ (note that
%% the domain and co-domain of $\hat{f}$ may be subsets of $X$ and $Y$, for example floating-point numbers are a subset of the real numbers).
%% The absolute error $\varepsilon_{abs}$ and relative error $\varepsilon_{rel}$  
%% are defined by
%% \begin{equation}
%% \varepsilon_{abs}(x) = | f(x) - \hat{f}(x)|\quad\text{and}\quad \varepsilon_{rel}(x)= \left| \frac{f(x)-\hat{f}(x)}{f(x)}\right|\enspace .
%% \end{equation}
%% The error $\varepsilon_{abs}(x)$ corresponds to the forward error. The backward error $\varepsilon_{back}(x)$ is defined as the distance
%% to the closest input $x'$ such that $f(x')=\hat{f}(x)$ or, in other word, the distance to the closest input
%% for which $\hat{f}(x)$ is the exact result.
%% \begin{equation}
%% \varepsilon_{back} = \min\ \big\{ x'\in X\ :\ f(x')=\hat{f}(x)\big\}\enspace .
%% \end{equation}
%% To our knowledge, all the existing tools measuring automatically errors on pieces of codes compute forward errors and,
%% in the following, we only consider this kind of error.

%% The object of the measure refers to what quantity is measured.  Let $\varepsilon(x)$
%% stand for either $\varepsilon_{abs}(x)$ or $\varepsilon_{rel}(x)$ depending on the kind of error we use. 
%% We may distinguish the following cases.
%% \begin{description}
%% \item[Worst case error] The worst case error $\varepsilon_{wst}$ is the error measured in the worst case for all the admissible inputs of the program. 
%% \begin{equation}
%% \varepsilon_{wst} = \max\ \big\{ \varepsilon(x)\ :\ x\in X \big\} \enspace .
%% \end{equation}
%% \item[Average error] The average error $\varepsilon_{avg}$ is the mean error measured for the considered inputs. Let Card$(X)$ denote
%% the cardinal of the set $X$.
%%  \begin{equation}
%% \varepsilon_{avg} = \frac{1}{\text{Card(X)}} \cdot \sum_{x\in X} \varepsilon(x) \enspace .
%% \end{equation}
%% \item[Worst anti-improvement] In general, accuracy optimizing tools cannot improve the accuracy of some computation for
%% all the acceptable inputs (apart by generating as many specialized versions of the code as needed). Consequently,
%% beside the worst case and average errors, other measures are relevant. The worst anti-improvement $\iota_{anti}$
%% measures the additional error added by the transformation in the worst case. Let $\hat{f}_{opt}$ be an accuracy optimized
%% version of $\hat{f}$. It is admitted that $\varepsilon_{wst}(\hat{f}_{opt})< \varepsilon_{wst}(\hat{f})$
%% or $\varepsilon_{avg}(\hat{f}_{opt})< \varepsilon_{avg}(\hat{f})$. Nevertheless, $\hat{f}_{opt}$ may be
%% less accurate than $\hat{f}$ on some input.
%% \begin{equation}
%% \iota_{anti}(\hat{f},\hat{f}_{opt}) = \max \left\{ \big| |f(x)-\hat{f}(x)| - |f(x)-\hat{f}_{opt}(x)|   
%% \big|\ :\ x\in X\right\}
%% \end{equation}
%% \item[Worst case and average improvement] The worst case improvement $\iota_{wst}$ is the improvement of error in the worst case. The average improvement $\iota_{avg}$ is the mean improvement of error measured 
%% for the considered inputs.  They are defined by
%%  \begin{equation}
%% \iota_{wst} = \varepsilon_{wst}(\hat{f}) - \varepsilon_{wst}(\hat{f}_{opt})\enspace,
%% \end{equation}
%%  \begin{equation}
%% \iota_{avg} = \frac{1}{\text{Card(X)}} \cdot \sum_{x\in X}  \big| |f(x)-\hat{f}(x)| - |f(x)-\hat{f}_{opt}(x)|  \big|\enspace .
%% \end{equation}
%% \end{description}


%% Note that other measures could be considered. For example, measures performing a trade-off between accuracy of the
%% results and performances of the code would be of great interest. Other criteria, like the impact of the transformation
%% on the numerical stability of the code could be taken into account. These measures could rely on the value
%% of the derivatives of the code obtained by automatic differentiation \cite{} or by the computation of Lyapunov exponents \cite{}.


%% \textbf{\large Work in progress below}

%% Independently of the object measured,
%% there exists two families of techniques to compute the errors based on static or dynamic analysis of the codes.
%% Dynamic techniques take a subset $X' \subseteq X$ of the inputs, in general by chosing random elements of $X$,
%% and compute $\varepsilon_{wst}$,  $\varepsilon_{avg}$, $\iota_{anti}$, $\iota_{avg}$, $\ldots$ using $X'$
%% instead of $X$. While dynamic techniques avoid to over-approximate the errors as it may happens when static 
%% techniques are used, the errors may also be under-approximated which may be an issue in some applicative
%% contexts such as critical embedded systems. Another difficulty of dynamic methods is the execution time of the
%% multiple runs, specially for programs with multiple inputs. In this case, the way random inputs are generated
%% must be specified. For example, for a program taking two inputs $x$ and $y$, one may take $n$ sample values for $x$ and
%% $n$ other values for $y$ and execute the $n\times n$ tests of the cartesian product or, alternatively, one may generate
%% a suite of $n$ pairs of data $(x,y)$ and execute the $n$ tests. Obviously, this impacts the execution time and the scalability of the tool
%% as well as how we may trust the results. Let us also remark that probabilities may be used in another ways, for example
%% as in the CESTAC method, implemented in the Cadna tool \cite{} in which each operation is run several times with random rounding modes
%% (towards plus or minus infinity with a probability of $0.5$).

%% Static analysis techniques compute an other-approximation of the errors without executing the programs. They rely
%% on non-relational \cite{} or relational abstract domains \cite{}. Static analysis techniques also use 
%% acceleration techniques (widenings) to over-approximate the results of programs with loops without executing all the iterations.
%% While this is another source of other-approximation, it makes it possible to obtain in a short time general results which hold for 
%% programs independently of the number of iterations.
\end{document}
